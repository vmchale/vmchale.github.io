\documentclass{article}
\usepackage{amsmath}
\begin{document}
\title{Jacobi's Principle}
\author{Thomas McHale}
\maketitle
We start with principle of least action, viz.
\begin{equation}
\delta\int_{t_1}^{t_2}L(q_i,\dot{q_i})dt=0
\end{equation}
for the path that the particle takes. In our case, we are going to introduce an arbitrary parameter $\tau$ and treat the time as one of the position co\"ordinates. Primes denoting differentiation\footnote{Actually, the $q_i'$ are independent of the $q_i$ but we add the auxiliary conditions $\frac{dq_i}{d\tau}=q_i'$.} with respect to $\tau$, we have
\begin{equation}
\delta\int_{\tau_1}^{\tau_2}\left(L\left(q_i,\frac{q_i'}{t'}\right)t'\right)d\tau=0
\end{equation}
Now we can apply the Euler-Lagrange equations for $t$ to this integral to get
\begin{equation}
\frac{\partial(Lt')}{\partial t}-\frac{d}{d\tau}\left(\frac{\partial(Lt')}{\partial t'}\right)=0.
\end{equation}
Since $t$ does not appear in the integrand, $\displaystyle\frac{\partial(Lt')}{\partial t}=0$, and thus $\displaystyle\frac{\partial(Lt')}{\partial t'}$ is constant with respect to $\tau$.

The generalized momentum $p_i$ for a variable $q_i$ will be defined as $\displaystyle\frac{\partial L}{\partial\dot{q_i}}$; in our case we also consider the momentum associated with time, viz. $\displaystyle p_t=\frac{\partial(Lt')}{\partial t'}.$

Now,
\begin{align*}
p_t&=\frac{\partial(Lt')}{\partial t'}\\
&=L-\left(\sum_{i=1}^n\frac{\partial L}{\partial\dot{q_i}}\frac{q_i'}{t'^2}\right)t'\\
&=\left(L-\sum_{i=1}^n\frac{\partial L}{\partial\dot{q_i}}\dot{q_i}\right)\\
&=\left(L-\sum_{i=1}^n p_i\dot{q_i}\right)\\
&=-E.
\end{align*}
where $E$ is the energy of the system.

Forming $\bar{L}=Lt'-p_tt'$, we will show that
\begin{equation}
\delta\int_{\tau_1}^{\tau_2}\bar{L}d\tau=0
\end{equation}

Now,
\begin{equation}
\displaystyle\delta\int_{\tau_1}^{\tau_2}p_tt'd\tau.
\end{equation}
will be true if and only if the Euler-Lagrange equations hold for all the variables. This is clearly true, since
\begin{equation}
\frac{\partial(p_tt')}{\partial q_i}-\frac{d}{d\tau}\left(\frac{\partial(p_tt')}{\partial q_i'}\right)=0
\end{equation}
$p_t$ being constant and the $q_i,q_i'$ being independent of $t'$. Also, 
\begin{equation}
\frac{\partial(p_tt')}{\partial t}-\frac{d}{d\tau}\left(\frac{\partial(p_tt')}{\partial t'}\right)=0
\end{equation}
since $\displaystyle\frac{\partial(p_tt')}{\partial t'}=p_t,$ which is constant.

We already know $\displaystyle\delta\int_{\tau_1}^{\tau_2}Lt'd\tau=0,$ so, adding the two, we get $\displaystyle\delta\int_{\tau_1}^{\tau_2}\bar{L}d\tau=0.$ From above, 
\begin{equation}
\bar{L}=Lt'-p_tt'=(L-p_t)t'=\left(\sum_{i=1}^np_i\dot{q_i}\right)t',
\end{equation}
so 
\begin{equation}
\int_{\tau_1}^{\tau_2}\bar{L}d\tau=2\int_{\tau_1}^{\tau_2}Tt'd\tau.
\end{equation}
It was pointed out by Jacobi that this cannot be simplified to $\displaystyle\int_{\tau_1}^{\tau_2}\bar{L}d\tau=2\int_{t_1}^{t_2}Tdt,$ because $t$ cannot be treated as an independent variable in the variational problem. Instead, we take advantage of the fact that
\begin{equation}
T=\frac{1}{2}\left(\frac{ds}{dt}\right)^2
\end{equation}
or
\begin{equation}
T=\frac{1}{2}\frac{\left(\frac{ds}{d\tau}\right)^2}{t'^2}
\end{equation}
Since $T=E-V,$
\begin{equation}
t'=\frac{1}{\sqrt{2(E-V)}}\frac{ds}{d\tau},
\end{equation}
giving finally
\begin{equation}
2\int_{\tau_1}^{\tau_2}Tt'd\tau=\int_{\tau_1}^{\tau_2}\sqrt{2(E-V)}\frac{ds}{d\tau}d\tau=\int_{\tau_1}^{\tau_2}\sqrt{2(E-V)}ds,
\end{equation}
as desired. Recalling that $\displaystyle\delta\int_{\tau_1}^{\tau_2}\bar{L}d\tau=0,$ we see that
\begin{equation}
\delta\int_{\tau_1}^{\tau_2}\sqrt{2(E-V)}ds=0
\end{equation}
as well. This condition determines the particle's path, and is known as Jacobi's principle. Note that this determines the path a particle takes in space, but says nothing about time.

Fermat's principle of least time states that light takes the path that takes the least time, that is,
\begin{equation}
\delta\int dt=0
\end{equation}
Rewriting $dt$ as $n ds$, $n$ being the index of refraction, we get
\begin{equation}
\delta\int nds=0.
\end{equation}
As can be seen, this bears a striking resemblance to Jacobi's principle. If a material has an index of refraction $n(x,y,z)$, then light will travel through it the same way a particle would be affected by a potential field $V(x,y,x)$ provided that $n=\sqrt{2(E-V)}.$ This is not the entirety of the optico-mechanical analogy, but it is part of it. The optico-mechanical analogy was used in the development of the old quantum theory, including de Broglie's Nobel Prize-winning dissertation on matter waves.

\end{document}
