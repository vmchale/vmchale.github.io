\documentclass{article}

\usepackage[T1]{fontenc}
\usepackage{verse}
\newcommand{\attrib}[1]{%
\nopagebreak{\raggedleft\footnotesize #1\par}}
\setlength{\leftmargini}{0em}
\usepackage{setspace}
\renewcommand{\baselinestretch}{1.1} 

\begin{document}

\title{Quelques \'Enigmes du livre d'Exeter}
\author{Vanessa McHale}
\maketitle

%\section{Pr\'eface}
\section{Notes}
Le num\'erotage des \'enigmes vient de l'\'edition Krapp-Dobbie du livre d'Exeter.

La po\'esie viel anglais accorde importance \`a versification allit\'erative, au lieu de rimes ou hexam\`etre dactylique. Pour ce raison, j'ai d\'ecid\'e de traduire en prose; je n'ai aucune experience \'ecrire avecs tels mod\'eles ni en anglais ni en fran\c{c}ais. 

Je n'ai pas pu trouver une traduction de ces \'enigmes en fran\c{c}ais, donc j'esp\`ere que ceci aidera quelqu'un. 

Mon fran\c{c}ais n'est pas le meilleur donc envoyez-moi un \'email si vous trouvez une faute. 

\poemtitle{Riddle 25}
\begin{verse}
Ic eom wunderlicu wiht\qquad     wifum on hyhte\\
neahbuend\=unyt\qquad     n\ae ngum sce\th\th e\\
burgsittendra\qquad     nym\th e bonan anum\\
sta\th ol min is steapheah\qquad     stonde ic on bedde\\
neo\th an ruh nathw\ae r\qquad     ne\th e\dh hwilum\\
ful cyrtenu \qquad     ceorles dohtor\\
modwlonc meowle\qquad     \TH  heo on mec gripe\dh\\
r\ae se\dh mec on reodne\qquad     reafa\dh min heafod\\
fege\dh mec on f\ae sten\qquad     fele\th sona\\
mines gemotes\qquad     se\th e mec nearwa\dh\\
wif wund\=en locc\qquad     w\ae t bi\dh \th\ae t eage.
\end{verse}
\attrib{Livre d'Exeter}

Je suis une chose magnifique,    une joie aux femmes,
utile aux voisins,    blesse aucune
des villageois    sauf seul 

\end{document}
