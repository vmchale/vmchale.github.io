\documentclass{letter}
\usepackage{verse}
\newcommand{\attrib}[1]{%
\nopagebreak{\raggedleft\footnotesize #1\par}}
\setlength{\leftmargini}{7em}

\title{Quelques \'Enigmes du livre d'Exeter}

\begin{document}

\section{Pr\'eface}
Le num\'erotage des \e'nigmes vient de l'\'edition Krapp-Dobbie du livre d'Exeter.

La po\'esie viel anglais accorde importance a mod\`eles des accents sur syllabes et allit\'eration, pas rimes. Pour ce raison, j'ai d\'ecid\'e de traduire en prose; je n'ai aucune experience \'ecrire avecs tels mod\'eles en anglais ou fran\c{c}ais. 

Je n'ai pas pu trouver une traduction de ces \e'nigmes en fran\c{c}ais, donc j'esp\'ere que ceci aidera quelqu'un. 

Mon fran\c{c}ais n'est pas le meilleur donc envoyez-moi un \'email si vous trouvez une faute. 

\poemtitle{Riddle 25}
\begin{verse}
Ic eom wunderlicu wiht\qquad     wifum on hyhte\\
neahbuendū nyt\qquad     nængum sceþþe\\
burgsittendra\qquad     nymþe bonan anum\\
staþol min is steapheah\qquad     stonde ic on bedde\\
neoþan ruh nathwær\qquad     neþeð · hwilum\\
ful cyrtenu ·\qquad     ceorles dohtor\\
modwlonc meowle\qquad     ꝥ heo on mec gripeð\\
ræseð mec on reodne\qquad     reafað min heafod\\
fegeð mec on fæsten\qquad     feleþ sona\\
mines gemotes\qquad     seþe mec nearwað\\
wif wundēn locc\qquad     wæt bið þæt eage.
\end{verse}
\attrib{Exter Book}

Je suis une chose magnifique,     une joie aux femmes,
utile aux voisins. 

\end{document}
