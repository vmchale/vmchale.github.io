\documentclass{book}

\usepackage[pagestyles]{titlesec}
\titleformat{\chapter}[display]
  {\normalfont\bfseries}{}{0pt}{\Huge}
\newpagestyle{mystyle}
{\sethead[\thepage][][\chaptertitle]{}{}{\thepage}}
\pagestyle{mystyle}

\begin{document}
\title{The Fifteen Joys of Marriage}
\author{A. de La Sale\\
Translated by Vanessa E. McHale}
\date{}
\maketitle
\tableofcontents

\subsection*{Translator's notes}
As of writing, there is no English edition of this work in print. I hope this will bring it to a somewhat wider audience. 

However, this is very much a work in progress. Future revisions will have more idiomatic English (and prettier \LaTeX). 

\chapter{The Seventh Joy of Marriage}
The seventh joy of marriage occurs when sometimes the one who who is married finds a very good wife, sage and well-behaved. Occasionally, it comes to pass that he finds a woman who is also very pleasant, who never denies the rationality that he offers her. Know that, no matter her temperment, honest woman or not, there is one general rule of marriage, which everyone believes and which holds universally: that her husband is the most incompetent and unskilled in the secret matter of all men in the world. Often, it happens that the inexperienced young man, infatuated, marries a good young woman and an honest woman, and they partake in pleasures together, as long as they can - one year, two years, three years, or more - until their youth fades; but the woman does not wither so fast as the man, no matter how strong and upstanding he may be:
\footnote{the original includes the word "gaste" which means both "Caractère de verticalité, de solidité" and "manière d'être (plus ou moins stable ou durable) d'une personne ou d'une chose." Hence this probably refers both to his physical state and his erection.}
this is because she does not endure the efforts, travails
\footnote{Again, probably a sexual pun. "Travaulz" can mean both "Activité, effort physique" or "Peine que l'on supporte, tribulation, souffrance, tourment que l'on endure"} 
, and worries which he does; and if she does nothing but amuse
\footnote{"soulacier" can mean both "Prendre du bon temps, se distraire" and "S'épanouir" - whether the latter implied orgasm is unclear.}
herself and play
\footnote{The word "jouer" may be a pun on "jouir"}
around, even then the man will be expired
\footnote{"Gasté" can refer to a state of deterioration, but "gaster" can also mean "to disperse" - in this context probably to ejaculate}
before her in this domain. It is indeed true that the woman, insofar as she bears children and is pregnant, is really hindered, and at childbirth has many pains and aches: but this is nothing in comparison to the concerns that a reasonable man entertains, or the deep thoughts he thinks in order to do the important things he has to do. And large though the pain of the pregnant woman may be, I do not marvel at her one bit more than a chicken or a wild goose who expels an egg the size of a fist - through an opening that you have never put a small finger in! It is not a big deal for nature to do one thing or the other: you will also see a chicken getting fatter than the rooster every day while laying eggs and a rooster remaining the same; for the rooster is so dumb that he does nothing the whole day but find her sustenance and putting it in her beak, while the chicken does not trouble herself with anything beyond eating, clucking 
\footnote{"caqueter" meant both "to cluck" and "to chat," which makes the metaphor work better}
, and taking it easy. So do good honest men do, for which they deserve praise. After the good man is frazzled and overwhelmed, all the while dealing with his pains, travails, and worries not to mention his other thoughts; he no longer applies himself to pleasures and diversions, and even more rarely to pleasing his wife; even worse, he cannot do it like he is used to, and he does not hold back anything
\footnote{i.e. he ejaculates prematurely}
whenever they are in this situation. This the woman does not do, instead she is as capable as at she was any time in the past. And it is for this that her value
\footnote{also "delivery" perhaps referring to the attainment of orgasm}
is diminshed every day. The pleasantries, the delights, the good appearances that were shared in youth and by capability
\footnote{capability here is in contrast to "impotence"}
of the husband, turn into arguments and feuds. Even worse, as her value fades little by little, they start to make sullen faces.

And when the woman's value no longer suffices, supposing she is a good honest woman, and that she did not want to do do any harm, if she does not let herself believe that her husband is more impotent than the others. She will have good reason to believe this since she has never tried any man but him, and he did not suffice her. By reason, one man must be enough for a woman, or nature wouldn't have proportioned things so well. In fact, I think that if one man was not enough for a woman, God and the Church would have commanded that everyone would have two, or as many as they needed. Sometimes, some even take themselves on the adventure of trying to find out if others are as impotent as their husbans. And after they go on this adventure they believe it even more than before, because in this adventure they take a companion that they can only meet
\footnote{"finer" also means "to finish"}
while scared and naked. Desperate as the situation is, it makes for marvels when it is allowed to happen. And if she had thought her husband unskilled and impotent before, she believes it even more strongly now, since the pleasures of the present are always worse than memories of those which have passed; hence she believes it more firmly than before, since experience is in control. 

Then, of course, the one who married finds a woman who is pleasant, and listens to the rationality that he tells her. She has the same opinion of her husband as the other wife: since on her adventure she has tried others, whose capabilities were far greater than those of the good man, who never put in much effort since he knew he would always find her near to him. 

Know that men are the exact opposite of what was just described, since, no matter which wife they have, they generally believe she is better than all the others. Sometimes the rule doesn't hold, but they are hopelessly debauched and incapable of reason or intelligence.
\footnote{the word "ribaux" can also refer to adulterous}
That is why we easily see many husbands praise
\footnote{"louent" can also be a pun on "rent out to others"}
their women, while telling of their good qualities; in their opinion there can be no one as good, where they can find so many good qualities, with such a high value, with such a good sexual appetite. So we often see that when a woman is widowed, she remarries without even waiting for another, sometimes not even waiting one month to see if the new one is as feeble and impotent as the one who is dead. And thus it happens that she treats him too with neither loyalty nor honesty. 

Often, the women who bears herself like this destoys everything by her poor behavior, wasting the things her husband worked so hard for (depending on the state he is in), spending in various manners, sometimes with her new friend, sometimes with an old pimp, or her priest who received the large payments to absolve her of her sins (many men want the power of the pope). And the good husband comports himself as sagely as he can, without making large purchases, has calculated the dividends, salary, or profits (depending on his social class) as well as his expenses. Then he finds, when all is counted and recounted, that things are not going well, and this gives him lots of worries. When he is retired
\footnote{Another sexual pun. The word "retrait" can mean "place where one retires" but also "place where fish are stored"}
he talks to his wife, who he loves more than himself, and says::

``Truly, my friend, I don't know what it is, but I don't where our money is going, be it wheat wine, or other thigns. For my part, I've always had an eye on our spending, to the point where I don't even dare buy myself new clothes''

``Truly, my friend, I'm as stunned as you are. I don't know what it could be either, since I am searching in vain to fix it and conduct things the best I can as cautiously as I can. ''


So the brave man has no idea what is happening, and comes into poverty, and does not know what to think, except to conclude that he alone is this unfortunate, and that it is Fortune that is plotting against him. He never believes anything that people say about his wife, nor does he ever find anyone who says anything about his wife, except by accident, because he would would be their biggest enemy. 

It happens sometimes that there is a good friend, who see the treatment he is getting, and can't help but say that he should watch out, or by accident will speak the truth, which would make stun him. Like this, he goes off and makes an unpleasant face, so that his wife knows well that something is wrong. She doubts herself and the adventure, because he has recently reproached her. But, if God wishes it, she well get by fine. The good man still says nothing, and thinks of how to try her:

``My friend, I have to go 30 miles away.''

``To do what, my friend?'' she says.

``I need to go somewhere for some things and for some others.'' he says. 

``I'd like it better, my friend, if you sent a valet.'' she says.

``I think that I will regret this, but I will return in two or three days.'' he says.

Then he departs, making it seem like he's going away, but he sets a trap. He puts himself in a place where, if anything happens in his house, he'll know. And the wife who realized what had been said to him, tells her friend not to come over for anything, because she suspects something.

So the wife conducts herself so gracefully that, with God watching, her husband would never find a fault. When the good man is done with his surveillance, he pretends to return to his house with a happy face, because he thinks it was just lies. It defies belief that the woman who made a happy face at him, who kissed and embraced him so tenderly, who called him "my love..." would ever be able to do such a thing. He sees that the rumors were nothing. When he is in private with his wife, he says:

``Truly, my love, people have told me some things that upset me.''

``By God, my love, I do not know what it is, but there was a good bit where you made such an unpleasant face. I was afraid you'd had a great loss, or that our friends had died, or taken by the English.''

``It's not that,'' he says ``it's even worse than what you say''

``Ave Maria'' she says ``And what could that be? I would be much happier if you told me.''

``Certainly. One of my friends reported that someone was having an affair with you, plus enough other things.''

At this point, the woman makes the sign of the cross and shows a look of great surprise, then smiles and says:

``My love, don't make any more unpleasant faces. By my fidelity, my love, I can only wish to be as free from all my sins as I am from this one.''

She puts a hand on his head and says:

``My love, I will not only swear on this, I will also give to the devil everything there is under my two hands, if once a man's mouth touched my lips that wasn't yours, your cousin's, or my cousin's when you told me too. Ugh! Was that all?'' she says. ``My love, I am glad you told me this, as I was worried it was something else. And I know well where these rumors came from. But, would to God, my love, that you know why he told you. By my faith, you would be quite surprised, when he says he's your friend. But deep down, I'm quite happy that he's rocked the boat.''

``And why did he do it?'' says the good man.

``Don't worry yourself, my love, you'll know another time.''

``Really,'' he says ``I want to know.''

``By God, my love, I was concerned that you were bringing him here so often and but I didn't trouble you by saying anything, because you told me you liked him so much.'' she says.

``Tell me'' he says ``I'm begging you.''

``Actually, my love, it is really not necessary
\footnote{here, "mestier" can mean both "necessary" or "sexual activity"}
 that you know right now.''

``Tell me'' he says, ``I want to know.''

So she kisses him and embraces him gently, and says:

``My love, my tender sire, why did they want to lower me in your eyes, the false traitors?''

``Tell me now, my love, what it is''

``By God, my love, that I love above all else on Earth, the traitor you trusted, who told you the rumors, begged me for more than two years to trick you: but I refused all the advances, and put forth many great efforts, in many ways; and when you were tricked into believing that he was coming here because he like you, he came only for betrayal; he did not want to stop, until I swore to him that I would tell you. But I could not bear to tell you, because it did not concern me, because I am sure of myself but did not want to create a quarrel between you, because I was deceived to think he would keep quiet. Alas! It is not his fault that he brought you shame.''

``Santa Maria!'' he says ``Is he really a traitor? I had never doubted him!''

``By God, daddy, if he ever again enters your house, or if ever I learn that you talked to him, I will never live in your company, because, by my faith, it is not me that you must watch. If God pleases, it is not now that I will begin; I pray to God with joined hands that at the hour he takes me he will take me with fire raining down from the sky when I have the desire, which will burn me alive. Alas! My very-tender love,'' she said while embracing him ``I would be a thousand times a traitor if I did hurt you, qhile you are so handsome, so good, so gentle, and so gracious, and want whatever I want. God would never tolerate that I live were I an adultress. And again, my love, je want and I beg you that you protect your house from those who accused me of being a traitor, My soul will be with the devil, if he once in my life made advances. But, in the name of God, I do not want him to come any place that I am.''

Then she begins to cry, and the good man appeases her, and promises and swears that he will guard everything she has told him to, that he will no longer forbid her young companion in the house, and he swears that he will never believe anything, or listen to anyone in the world. In any case, he will never be free of inner torment, and his heart is a little exhausted. At the end of it all, his love, who said this only for her own good, will shortly be his biggest enemy. Thus the proud man is reduced to a beast, eating pasture, transfigured into an animal without magic. Now he is trapped
\footnote{"nasse" also means fish trap, which may mean something}
in the household. The woman will do even better in this situation than she was every able before. But do not say this to the man, because he will never believe anything while he who tells him that he mistreated her will be the greatest friend that he can have. Old age will surprise him, and luck will watch him diminish in poverty, from which he will never recover. See here the pleasantness which is found of the trap of marriage! Everyone mocks hims: one says that it's a great shame, because he is a good man; the other says one cannot worry, and that it's just the rules of the game, and that he is nothing more than an animal. Notable citizens push him and quit his company. So he lives with pains and sorrows, which he takes for joys, in which he will remain forever, and finish his days miserably. 

\end{document}
