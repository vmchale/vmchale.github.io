\documentclass{article}
\usepackage{amsmath}
\usepackage{amsfonts}

\begin{document}

\title{Noether's Theorem for Functionals Depending on Higher-Order Derivatives}
\author{Vanessa E. McHale}
\maketitle

\begin{abstract}
Presents proofs of some theorems involved in the calculus of variations with fields depending on higher order derivatives. We follow Gelfand and Fomin very closely.
\end{abstract}

\section{The Euler-Lagrange Equations in general}

Suppose we have a functional of the form $J[u]=\int F(x_i; u_j;\frac{\partial u_j}{\partial x_i};\frac{\partial^2u_j}{\partial x_j x_k};\cdots)dx_1\cdots dx_n$. We wish to find a sufficient condition that it is stationary.

\section{Calculation of $\delta u_{x_i x_j}$}

We get that $$(\bar{\delta u})_{x_ix_j}+\displaystyle\sum_{j,k=1}^nu_{x_i x_j x_k}\delta x_k$$

\section{General Expression for the variation of a functional}

\section{Conserved flows}
Let us suppose we are given a functional $J[u]$ which is invariant under a transformation of the form

$$x_i^*=\Phi_i(x,u,\partial_iu,\partial_i\partial_ju;\epsilon)~x_i+\epsilon\phi_i(x,u,\partial_i u)$$
$$u_j^*=\Psi_i(x,u,\partial_iu,\partial_i\partial_ju;\epsilon)~x_i+\epsilon\psi_j(x,u,\partial_i u)$$

that is, $\int F^*(u*)dx^*=\int F(u)dc$.

Then 

$$\sum_{i=1}^n\frac{\partial}{\partial x_i}M=0$$

whenever $u_j$ are chosen to be extremal.  

\end{document}
